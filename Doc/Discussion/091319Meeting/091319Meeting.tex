% acex13.tex/06/24/2013\documentclass[mathserif]{beamer}%
\documentclass{beamer}
\usepackage{amsxtra,amssymb,amsthm,amsmath,latexsym}
\usepackage{multicol}
\usepackage{beamerthemesplit}
%\usetheme{Darmstadt}
\numberwithin{equation}{section}
\newtheorem{thm}{Theorem}[]
\newcommand{\EN}{\mathcal{N}}
\newcommand{\na}{\nabla}
\newcommand{\R}{{\mathbb R}}
\newcommand{\ra}{\rightarrow}
\newcommand{\ue}{\infty}
\newcommand{\nc}{\newcommand}
\newcommand{\RRR}{\mathbb{R}^3}
\newcommand{\ol}{\overline}
\newcommand{\thmref}[1]{Theorem~\ref{#1}}
\newcommand{\lemref}[1]{Lemma~\ref{#1}}
\newcommand{\xy}{\boldsymbol{x}}
\newcommand{\uv}{\boldsymbol{u}}
\newcommand{\conv}{\circledast}
\newcommand{\llangle}{\left\langle}
\newcommand{\rrangle}{\right\rangle}
\def\bee{\begin{equation*}}
\def\eee{\end{equation*}}
\def\be{\begin{equation}}
\def\ee{\end{equation}}
\nc{\Ba}{\Big(} \nc{\Bz}{\Big)} \nc{\pa}{\partial} \nc{\ti}{\times}
\nc{\n}{|} \nc{\al}{\alpha} \nc{\da}{\delta} \nc{\bs}{\backslash}
\nc{\ka}{\kappa} \nc{\Da}{\Delta} \nc{\si}{\sigma} \nc{\f}{\big(}
\nc{\g}{\big)}
%\nc{\n}{|}
\nc{\om}{\omega}
\begin{document}
%%%% Test - Einsatz  Kapitel 5 -
\part{ }
%XXXXXXXXXXXXXXXXXXXXXXXXXXXXXXXXXXXXXXXXXXXXXXXXXXXX
\title[CIRL Discusson]{9/13/19 Discussion}
\institute{}
\date{}
\frame{\titlepage}
\section{Recap}
\frame{
\begin{itemize}
\item[1.] We can run the model-based methods for the 3W system, using the on-the-fly up-sampling and computation of the patterns, without using the GPU. It takes about 1 and a half day for 200 iterations for the model-based method and about 3 days for 200 iterations for the model-based method with positivity constraint. Using GPU is not stable and so not suggested at this time.
\item[2.] New multi-object.
\item[3.] Parameters and PSF for FairSIM data.
\end{itemize}
}

\section{Fall 2019 plan}
\frame{
First task: investigate 3D-SIM data/restoration with simulated noise and noisy parameters, on a simulated bead; learn how to divide a task into smaller sub-tasks.
\begin{table}[First task]
\begin{tabular}{l|l|l|}
\cline{2-3}
      &  Cong             & Saba 				 \\
\hline
 9/13 &                   & Forward model 		 \\
 9/20 &                   & Down-sampling 		 \\
 9/27 & Conf. paper draft & MB reconstruction 	 \\
10/04 &        			  & Draft report 		 \\
10/11 & Conf. paper 	  & Noisy data           \\
10/18 &                   & \textbf{First report} - Noisy parameter      \\
10/25 &            		  & Noisy reconstruction \\
11/01 & FairSIM           & Analysis/Recap       \\
\hline
\end{tabular}
\end{table}
}

\frame{
Second task: investigate 3D-SIM data/restoration with noisy PSF (both simulated and experimental), on a more challenging simulated object. Saba will come up with sub-tasks.
\begin{table}[Second task]
\begin{tabular}{l|l|l|}
\cline{2-3}
      &  Cong               & Saba 				 \\
\hline
11/08 &                     & 					 \\
11/15 & AD/different Nslits & Noisy multi-object \\
11/22 & 				    &   				 \\
11/29 &        			    & (thanksgiving)	 \\
12/06 & AD results   	    & \textbf{Final report}       \\
\hline
\end{tabular}
\end{table}
}

\section{Next tasks}
\frame{
\begin{itemize}
\item[1.] Write the draft sections 2 (MBPC method), 4 (analysis of simulated results) of the ISBI paper.
\item[2.] Running the following experiments for the refined multi-object with refined axial frequency, for the 3W system, for both MB and MBPC methods:
\begin{table}[]
\begin{tabular}{c|cc}
      & Noiseless  & SNR = 15 dB \\ \hline
15/15 & 200 iter   & 200 iter    \\
 7/15 & 400 iter   & 400 iter    \\
 5/15 & 400 iter   & 400 iter    \\
\end{tabular}
\end{table}
\item[3.] FairSIM PSF and trying to get reconstruction for FairSIM data.
\end{itemize}
}

%% #
%\frame[shrink=30]{\frametitle{Test}
%Test
%}

\end{document}
